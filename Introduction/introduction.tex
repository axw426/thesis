\chapter{Introduction}

With the expected shutdown of the \ac{HL-LHC} in 2038, and the long time scales associated with the construction of any new colliding facility ($\sim$10 years), the time for physicists to agree on what experiments should follow in the post \ac{LHC} era is rapidly approaching, with initial decisions expected to take place in the early 2020s. However, following the discovery of a Higgs Boson at the \ac{LHC} \cite{:2012gk,Chatrchyan:2012xdj}, with properties in agreement with those predicted by the \ac{SM}, the particle physics community is left in a situation where there is no definitive course of action through which new physics phenomena might be discovered. There are still many open questions remaining; one clear example being the origin of dark matter, which has been observed to make up $\sim$27\% of the universe. Despite being examined through multiple astrophysical observations such as gravitational lensing or galaxy rotation curves \cite{Trimble:1987ee}, there is still no proven particle physics explanation for what it is made from or how it is produced. Other notable examples include the matter-antimatter aysmmetry of the universe which is yet to be explained by the levels of CP violation measured in the \ac{SM} and the Higgs hierarchy problem / fine tuning problem where it is expected that a precise cancellation of quantum corrections is needed to be able to simultaneously explain the difference in strength between the weak and gravitational forces while allowing for the measured value of the Higgs mass. Currently there is no clear direction for how we might solve these mysteries. As such, there are two main approaches that may be taken. The first is to continue to push the boundaries of the ``energy frontier'' and look for new physics phenomena at higher energy scales that are not predicted by the \ac{SM}. In this scenario the natural option is to build a circular hadron collider, much like the \ac{LHC}. While hadron collisons result in more complex interactions due to the fact they possess substructure, they are well suited for high energy collisions due to their high masses which reduce the amount of synchotron radiation emitted (radiation produced from accelerating a charged body through an \ac{EM} field (see \refeq{Eq:synchotron radiation}) when accelerating them in a circular path.

\begin{equation}
\label{Eq:synchotron radiation}
P = \frac{e^4}{6\pi\varepsilon_0m^4c^5}E^2B^2.
\end{equation}
Where $P$ is power, $e$ is elementary charge, $E$ is particle energy, $B$ is magnetic field, $m$ is mass and all other symbols have their usual meaning.

Pushing the energy frontier has the appeal that it allows direct detection of particles at new energy scales and is supported by the fact that many \ac{BSM} models rely on new particles appearing in the $>$multiTeV energy range e.g. supersymmetry, however it does have drawbacks and risks. Due to the composite structure of hadrons they provide collision energies that are often significantly below the provided beam energy and are challenging to measure. This limits the type of measurement that can be performed as the initial state of the interaction is poorly defined and so all measurments must rely on measurement of the final state particles. This increases the effect of uncertainties intorduced by detector accpetances and resolutions and makes it highly challenging to identify particles that can't be directly seen by the detector e.g. neutrino/ dark matter candidates. Due to fragmentation of the hadrons, there are also significant \ac{QCD} background jets produced which can dominate over potential new signal channels. While these do make measurements more challenging, the real risk with pushing the energy frontier is that the constraints on the scale of at which new physics might be observed are currently very poor. This makes it challenging to choose what collision energy any future collider should operate at as choosing too low an energy could result in no new phenomena being seen.

The second option is to advance in the ``precision frontier'' and search for small deviations from the \ac{SM} or harder to detect processes. In this case the more natural choice is to use a lepton collider as the fact the interaction is an annihilation of fundamental particles rather than composite particles means that the initial conditions of the interaction can be known to a high precision determined entirely by the quality of the colliding beams. For leptons it is also possible to produce polarised beams which opens up a new range of potential measurements when examining interactions that couple differently to left and right handed particles. Doing this, areas of the \ac{SM} that are less well measured such as the Higgs and Top quark sectors can be probed for evidence of physics beyond the \ac{SM}. The worst case scenario for a lepton collider is to simply reinforce the \ac{SM} without seeing any new phenomena, however even in this case the new levels of precision on many of the \ac{SM} parameters will be beneficial for constraining \ac{BSM} theories and reducing systematic uncertainties on measurements being made at other future colliders. While allowing for precision measurements of the \ac{SM}, lepton colliders do also provide opportunities for both direct and indirect discoveries of new physics through channels that are either unavailable at hadron colliders or that are challenging due to the \ac{QCD} backgrounds. The main draw back of colliding leptons is that currently the only viable option is to use electrons and positrons (though there is effort underway to use muons \cite{Bogomilov:2017vwz}) which have extremely low masses and so produce considerable levels of synchotron radiation (${10^{16}}$ as much as protons) when used in a circular collider. The usual solution to this is to use a linear collider instead. This prevents losses from synchotron radiation, however it limits the maximum collision energy that can be achieved as the path over which the particles can be accelerated is limited to the length of the accelerator which is itself limited by the increasing cost of extending the footprint of the machine. It is worth noting however, that for leptons the collision uses the full beam energy each time and so higher energy interactions can be produced from lower energy beams than for hadrons.

Considerable work has already been carried out into designing both high energy and high precision colliders. On the high energy side is the \ac{FCC}, a 100~TeV circular proton collider proposed as a project for \ac{CERN}. It is possible to also use this device as an $e^+e^-$ collider operating above the Higgs threshold so as to act as a ``Higgs factory''. On the precision side there are multiple proposed projects, however the most mature of these are the linear electron-positron colliders: \ac{CLIC} \cite{CLIC:2016zwp} and \ac{ILC} \cite{Behnke:2013xla}. The \ac{ILC} is a 500~GeV collider proposed as a joint endeavour between the Japanese government and the international community while \ac{CLIC} is a multi-TeV machine being proposed by \ac{CERN}. Due to the large cost of these devices it unlikely that \ac{CERN} would build both \ac{FCC} and \ac{CLIC} together.

The focus of this thesis will be on the prospects of the proposed high precision colliders. In particular we discuss the prospects for measuring properties of the Higgs Boson and Top quark at \ac{CLIC}, which are both relatively less precisely measured when compared to the other particles of the \ac{SM}, while also examining a novel design for a digital calorimeter based on \ac{CMOS} \ac{MAPS} technology for use in future detectors as an extremely high granularity \ac{ECAL}.  


 
