\chapter{INTRODUCTION}

Following the discovery of a Higgs Boson at the \ac{LHC} \cite{:2012gk,Chatrchyan:2012xdj}, with properties in agreement with those predicted by the \ac{SM}, the particle physics community is left in a situation where there is no definitive course of action through which new phsyics might be discovered. There are many open questions remaining such as the nature of dark matter or the origins of matter-antimatter asymmetry, but no clear direction for how to answer them. As such there are two main approaches that may be taken- the first would be to continue to push the boundaries of the ``energy frontier'' (following the approach of the \ac{LHC}) and look for new physics at higher energy scales that is not predicted by the \ac{SM} but is predicted by many \ac{BSM} models such as supersymmetry. The second option is to advance in the ``precision frontier'' to search for small deviations from the \ac{SM} and harder to detect processes. Both approaches come with their own advantages and disadvantages. Going to higher energies is more likely to allow direct detection of new particles and is supported by the fact that the majority of \ac{BSM} models predict that new physics effects should exist at energies beyond what has been explored so far, however the problem is that the scale at which new physics should appear is unknown. This makes designing a future high energy collider difficult as without a clear idea of what energy is needed it is possible the collision energy will be below the new physics scale and so no new phenomenum will be observed. Pushing the precision frontier has the drawback that it may only serve to reinforce confidence in the \ac{SM} and has a smaller chance of direct discovery of new physics, however even in this worst case scenario it will still provide precise measurements of \ac{SM} properties which are beneficial both for constraining \ac{BSM} models and for reducing uncertainties on future measurements at high energy colliders.

The choice in which the community decides to proceed will greatly influence the design options for the next generation of particle colliders. For high energy measurments the obvious choice will be a circular hadron collider similar to the \ac{LHC}. The circular design allows for particles to be accelerated over as large a time as necessary without the need for a large collider which facilitates reaching higher energies. It also allows for collisions to occur at multiple sites simultaneously allowing for more experiments to take place at the facility. For a circular design, hadrons (most likely protons due to their stability and charge) are needed as their high mass minimises energy loss from synchotron radiation. This is radiation produced by a charged particle undergoing a transverse acceleration and is described by equation \refeq{Eq:synchotron radiation}:

\begin{equation}
\label{Eq:synchotron radiation}
P = \frac{e^4}{6\pi\varepsilon_0m^4c^5}E^2B^2.
\end{equation}
Where $P$ is power, $e$ is elementary charge, $E$ is particle energy, $B$ is magnetic field, $m$ is mass and all other symbols have their usual meaning.

While protons allow for high energy collisions, they are not suitable for precision measurements as their composite nature means that the initial four monentum and quantum numbers of the collision cannot be known. This means that all information must be extracted purely from a collisions decay products which are subject to uncertainties from detector resolutions/acceptances and the decay products visibility e.g. neutrinos/dark matter cannot be detected. For precision physics the better choice is to collide electron-positron pairs. This is an annihilation reaction where all quantum numbers will cancel out and the initial energy and momentum are only limited by the quality of the colliding beams. This allows conservation laws to be used to infer the properties of missing particles such as neutrinos or new particles. For colliding electrons, a circular collider is no longer feasible as the electrons low mass result in energy being lost through synchotron radiation at ${10^{16}}$ times the rate of protons. Instead, lepton colliders are traditionally built as linear colliders. This prevents synchotron radiation occuring but limits the maximum collision energy achievable as the path over which a particle can be accelerated is limited to just one length of the collider and it is expensive and impractical to produce longer colliders. That being said, because a lepton collider provides an annihilation interaction rather than a parton interaction it is possible to build a lepton collider with a lower beam energy but still have a higher average collision energy.

With the \ac{HL-LHC} expected to finish taking data in the 2030s and the long construction times associated with super colliders, a decision on what form the next generation of colliders should take is expected to occur by the early 2020s. Considerable work has already been carried out into designing both high energy and high precision colliders. On the high energy side is the \ac{FCC}, a 100~TeV circular proton collider propsed as a project for CERN. On the precision side there are multiple proposed projects (REFERENCE SMALLER PROJECTS), however the most mature of these are the linear electron-positron colliders: \ac{CLIC} \ref{CLIC:2016zwp} and \ac{ILC} \ref{Behnke:2013xla}. The \ac{ILC} is a 500~GeV collider proposed by the Japanese government while \ac{CLIC} is a multi-TeV machine being proposed by CERN. Due to the large cost of these devices it unlikely that CERN will build both \ac{FCC} and \ac{CLIC} together.

The focus of this thesis will be on the prospects of the proposed high precision colliders. In particular we discuss the prospects for measuring properties of the Higgs Boson and top quark at \ac{CLIC} which are relatvely poorly measured when compared to other standard model particles, while also examining a novel design for a digital electromagnetic calorimeter based on \ac{CMOS} \ac{MAPS} technology for use in future detectors.  


 