\chapter{Theory}

Two studies are presented within this thesis for prospective measurements at CLIC looking at the $H\rightarrow WW$ branching ratio and the forward-backward asymmetry in \ttbar~ production at 1.4~TeV. As such it is important to first examine the significance of these measurements in the context of the wider physics programme of CLIC to understand their importance.


\section{Higgs}

Plot of Higgs production cross sections vs energy

\begin{figure}
  \centering
  \includegraphics[width=0.75\textwidth,height=10cm,keepaspectratio]{Experiments/fig/dummy}
  \caption[Cross Sections for Higgs Production Mechanisms as a Function of Collision Energy]{Cross Sections for Higgs Production Mechanisms as a Function of Collision Energy}
  \label{fig:higgsXSecs}
\end{figure}

The CLIC physics programme has a large focus on characterising the higgs boson due to the large uncertainties on many of it's associated properties relative to other sectors of the standard model. Electron positron collisions provide access to numerous higgs production mechanisms which can be seen in \ref{fig:higgsXSecs}. 

\subsection{HiggsStrahlung}

One of the key aims of the experiment will be to examine the HiggsStrahlung process shown in (FIGURE!!). In this process, if the Z boson decays to something easily reconstructable such as a pair of muons, then because the conditions of the collision are well known the four momentum of the higgs boson can be calculated from energy and momentum conservation. This allows properties such as the higgs mass, cross-section and coupling to the Z to be measured without actually ever measuring the decay products of the higgs boson which in turn allows the measurements to be model independent as few assumptions must be made about the interactions of the higgs. This method is not possible at hadron colliders such as the LHC where, even though the HiggsStrahlung process still occurs, the initial state can never be measured accurately due to the composite nature of the colliding particles. 

Show plots for the mass and width and quote the limits on the branching ratio of the higgs to invisible.


\subsection{Model Independent Extraction of Higgs Couplings}


While the HiggsStrahlung alone allows the mass, width and branching ratios of the higgs to be measured in a model independent manner, one can also extract model independent measurements of the higgs couplings by looking at other higgs production mechanisms. The extraction of the coupling requires a minimum of four measurements to be performed. One such ``recipe'' proposed for doing this is shown in FIGURE

Add in formulas for calculating model indepenent couplings

State why each step is chosen (high statistics) 

Highlight that the H-->WW measurment is necessary

Show the expected precision predicted on all the couplings, compare to model dependent measurements and to LHC




Explain that measuring the H->WW cross-section is key for model independent measurements of the Higgs. Justifies showing all subsequent Higgs measurements that rely on it!

Model independent measurements
Mass
Width
Couplings

\section{Top}

Define what the forward backward asymmetry is, brief history of measurents elsewhere

How it can be used to measure EW couplings

New physics that can effect these couplings e.g. Z'



AFB
EW Couplings
%Mass
%Width


