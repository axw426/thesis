\chapter*{ABSTRACT}
%
Within the next few years a decision must be made by the global community on what type of high energy colliders should be built in the post LHC era. Here we present studies showing what might be achieved if a linear lepton collider such as \ac{CLIC} is chosen. Two physics studies are presented showing the precision achievable in the electroweak sector when operating at 1.4 TeV. Firsty the measurement of $\sigma_{H\nu\bar{\nu}} \times BR(H\rightarrow WW^*)$, an integral component for model independent Higgs measurements, is described using the semileptonic decay channel and is shown to yield a statistical precision of 1.3\% for 1.5 ab$^{-1}$ of data. A differential measurement of the top quark forward backward asymmetry is also performed as a probe of the electroweak form factors of the ttX vertex yielding a statistical precision of $\mathcal{O}$(1\%) for 1.5 ab$^{-1}$ of data. Lastly, the potential for using a novel design of a \ac{DECAL} at the \ac{ILC} is studied showing that an energy resolution of $\frac{\sigma_E}{E}=\frac{16.1\%}{\sqrt{E}} \oplus \frac{0.5\%}{E} \oplus 0.4\%$ can be achieved, similar to what is seen for the standard design choice, when using 30 $\mu m$ pitch pixels with a 12 $\mu m$ epitaxial thickness.

%
\clearpage
\chapter*{DECLARATION OF AUTHORS CONTRIBUTION}
%
The work presented here has been carried out within the \ac{CLIC} and \ac{CALICE} collaborations however all the work presented here is solely the authors own work unless otherwise stated. The two physics studies presented in Chapters \ref{Higgs Analysis} and \ref{chapter:topanalysis} were performed using the ILCSoft framework used for all analyses at \ac{ILC} and \ac{CLIC} and used event samples generated centrally by \ac{CLIC}, however the reconstruction and event selection techniques used were chosen and implemented by the author. The study related to the Higgs sector has already been published in a paper summarising the potential for Higgs measurements\cite{Abramowicz:2016zbo} at \ac{CLIC} and a similar paper showing the potential for top quark measurements that will include the results from Chapter \ref{chapter:topanalysis} is currently under review\cite{TopPaperDraft}.

For the work presented in Chapter \ref{sect:DECAL} the DigiMAPS package was used for adding additional levels of realism to the simulation studies. This package was originally developed by Anne-Marie Magnan as part of the \ac{CALICE} collaboration and was adapted by the author to change how pixel noise was implemented and to allow for variations in the threshold between pixels. This work is in the process of being written into a paper by the author.

%
\clearpage
\chapter*{ACKNOWLEDGEMENTS}
%
Science is hard, PhDs especially so. People think you can just lie under apple trees until you get hit on the head and see the world anew but that's a once in a generation thing. In reality the apple just leaves you dazed and confused with a bit of a sore head. For the rest of us, being a scientist means chiseling away at the cold opaque walls of human knowledge in the hope of finding something worthwhile with no knowledge of whether we're even digging in the right place. It's not a glamorous or quick affair. It is a slow frustrating process that often yields little result. That being said, without it we don't progress. We wouldn't have the internet, anitbiotics or even things like tv and film. Without it we won't ever take our first step on a different planet, we won't cure cancer and we won't have that which we can't even imagine yet. As a result I'm proud to have been a part of it. Even if this research never comes to anything I can be proud that I've been able to carve a tiny ``AW was here'' in the void of human ignorance. As such I'd like to thank all those who have helped me be a part of the process either by showing me how to progress or by keeping me from giving in to the frustration.  

First and foremostly I need to thank my supervisor, Nigel Watson, for agreeing to take on a man with little to no particle physics or programming knowledge to do a PhD which is based entirely on knowing particle physics and being able to programme. His dedication to always find time for everyone despite having a virtual army of students to deal with and his ability to always explain things in a way even I could understand is something I am truly grateful for. Similarly I would like to thank Tony Price for all his help along the way even when he was no longer even part of the department and had no obligation to do so. I would also like to show my appreciation to all the member of the CLICdp working group for the valuable insight they have provided on both the analyses I worked on.

While the people above have all helped me develop as a physicist I would also like to thank those who have kept me sane and made the office a fun place to be: Andy ``legs'' C, Richard, Matt, Mark, Andy F, Jam\'{e}s, Tim Tim Tim, Daniel ``The Brigadier'', Elliot, James, Jack, John, Rob, Russell, Dan, Gov, Nandish and Robbie. While there have been many great moments over the years I think some of the highlights must certainly be the creation of danger juice, snail racing in the office and the double success in the Bubble Chambers football tournament. 

Special mentions also go to the various eateries that have kept me fuelled throughout the PhD. In particular, Go Mex for providing an unhealthy amount of burritos to my diet, Dilshad for making the best curry in Birmingham and The Goose for simply being The Goose.

Lastly I want to thank my wife Alison for moving to Birmingham in the first place, her tolerance of my moaning about work when she's just finished several night shifts nursing and for all her love and support through all the years. 
%
\cleardoublepage
~
\vspace*{\fill}
\begin{center}
  \emph{Motto or dedication}
\end{center}
\vspace*{\fill}
\cleardoublepage
